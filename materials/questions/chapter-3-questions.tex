\documentclass[11pt,a4paper]{article}
\usepackage[utf8]{inputenc}
\usepackage[margin=1in]{geometry}
\usepackage{amsmath}
\usepackage{amssymb}
\usepackage{enumitem}
\usepackage{tikz}
\usepackage{graphicx}
\usepackage{multirow}
\usepackage{array}
\usepackage{tabularx}
\usepackage[table]{xcolor}

\title{\textbf{Management in Global Markets\\Chapter 3: Practice Questions}}
\subtitle{International Business Strategy}
\author{Universidad Complutense de Madrid\\Group A2}
\date{Academic Year 2025-2026}

\begin{document}

\maketitle

\section*{Instructions}
This document contains practice questions covering Chapter 3: International Business Strategy. Use these questions to test your understanding of market entry, competition, and global business operations.

\vspace{0.5cm}
\hrule
\vspace{0.5cm}

\section{Market Entry Strategy}

\subsection{Fill-in-the-Blank Questions}

\begin{enumerate}[leftmargin=*]
    \item When moving a company from one country to another, you face multiple types of risks including \underline{\hspace{3cm}}, \underline{\hspace{3cm}}, \underline{\hspace{3cm}}, \underline{\hspace{3cm}}, and \underline{\hspace{3cm}} risks.
    
    \item Key indicators for market attractiveness include \underline{\hspace{3cm}} growth, \underline{\hspace{3cm}} growth, and \underline{\hspace{4cm}} risk index.
    
    \item \textbf{Country Risk} measures how \underline{\hspace{3cm}} a country is and whether the government might change, economy crash, or wars happen.
    
    \item \textbf{Trade Barriers} (tariffs) are taxes the government puts on \underline{\hspace{3cm}} stuff, making ingredients cost more.
    
    \item \textbf{Industry Rivalry} refers to how many \underline{\hspace{4cm}} are already in the market.
    
    \item The strategy of going where \underline{\hspace{4cm}} are validates the market and provides learning opportunities.
    
    \item \textbf{Localization} is necessary before \underline{\hspace{5cm}} can occur.
    
    \item In international business, you cannot escape \underline{\hspace{3cm}} and \underline{\hspace{4cm}}, but you can improve taxes, logistics, and finance.
\end{enumerate}

\subsection{Multiple Choice Questions}

\begin{enumerate}[resume, leftmargin=*]
    \item When choosing between Market B (many competitors) and Market C (few competitors), you should choose:
    \begin{enumerate}[label=\alph*)]
        \item Market C because less competition
        \item Market B because competitors validate the market
        \item Neither, find a new market
        \item Both simultaneously
    \end{enumerate}
    
    \item What is the primary reason to go where competitors are?
    \begin{enumerate}[label=\alph*)]
        \item To avoid market research
        \item Market validation and learning opportunities
        \item To copy their strategies exactly
        \item To reduce costs
    \end{enumerate}
    
    \item High tariffs affect business by:
    \begin{enumerate}[label=\alph*)]
        \item Making imported ingredients more expensive
        \item Increasing food costs
        \item Reducing customer base
        \item All of the above
    \end{enumerate}
    
    \item Market attractiveness indicators include:
    \begin{enumerate}[label=\alph*)]
        \item GDP growth, population growth, political risk
        \item Only GDP growth
        \item Only cultural factors
        \item Only competition level
    \end{enumerate}
\end{enumerate}

\section{Competition Strategy}

\subsection{Short Answer Questions}

\begin{enumerate}[resume, leftmargin=*]
    \item Explain why you should go where competitors are rather than avoiding them. What are the benefits?
    
    \vspace{4cm}
    
    \item Describe what happens when companies try to avoid competition. Use the real example from the course.
    
    \vspace{4cm}
    
    \item What does it mean to "differentiate" your offering in a competitive market? Provide examples.
    
    \vspace{4cm}
\end{enumerate}

\section{Product Quality and Country Reputation}

\subsection{Application Questions}

\begin{enumerate}[resume, leftmargin=*]
    \item Compare Japan and China in terms of product quality reputation. What are the implications for pricing?
    
    \vspace{4cm}
    
    \item Explain the relationship between quality and price in international markets. How do customers respond?
    
    \vspace{4cm}
    
    \item Why is country reputation important in international business? Give examples.
    
    \vspace{3cm}
\end{enumerate}

\section{Global Supply Chain}

\subsection{Case Study Questions}

\begin{enumerate}[resume, leftmargin=*]
    \item Explain the concept of containerization. Why are there different container sizes and types?
    
    \vspace{4cm}
    
    \item Describe the global manufacturing process using Boeing as an example. Why are parts sourced from multiple countries?
    
    \vspace{4cm}
    
    \item What are the key components of supply chain management? How do they work together?
    
    \vspace{4cm}
\end{enumerate}

\section{Risk Assessment}

\subsection{Analysis Questions}

\begin{enumerate}[resume, leftmargin=*]
    \item List and explain the different types of risks when entering a new international market. How can each be managed?
    
    \vspace{5cm}
    
    \item Create a risk assessment framework for evaluating a new market. Include all relevant factors.
    
    \vspace{5cm}
    
    \item Explain the difference between market risk and country risk. Provide examples.
    
    \vspace{4cm}
\end{enumerate}

\section{Market Entry Simulation}

\subsection{Synthesis Questions}

\begin{enumerate}[resume, leftmargin=*]
    \item Design a market entry strategy for a restaurant chain expanding internationally. Include:
    \begin{itemize}
        \item Market selection criteria
        \item Risk assessment
        \item Competition analysis
        \item Entry mode considerations
    \end{itemize}
    
    \vspace{8cm}
    
    \item Write an essay (400-500 words) explaining the complete process of internationalization, from initial market research to successful market entry. Include all key steps and considerations.
    
    \vspace{10cm}
    
    \item Compare and contrast different market entry modes (exporting, licensing, joint venture, direct investment). When would you use each?
    
    \vspace{6cm}
\end{enumerate}

\section{International Business Operations}

\subsection{Comprehensive Questions}

\begin{enumerate}[resume, leftmargin=*]
    \item Explain the concept of "glocalization" in the context of international business strategy. Provide examples.
    
    \vspace{4cm}
    
    \item How do cultural factors influence market entry decisions? Use specific examples from the course.
    
    \vspace{4cm}
    
    \item Create a comprehensive checklist for companies considering international expansion. Include all critical factors.
    
    \vspace{6cm}
\end{enumerate}

\newpage

\section*{Answer Key}

\textit{Note: Answers should be reviewed with course materials and instructor guidance.}

\subsection{Definitions}
\begin{enumerate}
    \item cultural, legal, logistical, financial, market, political (any 5)
    \item GDP, population, political
    \item unstable
    \item imported
    \item competitors
    \item competitors
    \item internationalization
    \item culture, communication
\end{enumerate}

\subsection{Multiple Choice}
\begin{enumerate}[resume]
    \item (b) Market B because competitors validate the market
    \item (b) Market validation and learning opportunities
    \item (d) All of the above
    \item (a) GDP growth, population growth, political risk
\end{enumerate}

\end{document}
