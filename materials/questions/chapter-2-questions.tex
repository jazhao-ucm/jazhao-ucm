\documentclass[11pt,a4paper]{article}
\usepackage[utf8]{inputenc}
\usepackage[margin=1in]{geometry}
\usepackage{amsmath}
\usepackage{amssymb}
\usepackage{enumitem}
\usepackage{tikz}
\usepackage{graphicx}
\usepackage{multirow}
\usepackage{array}
\usepackage{tabularx}
\usepackage[table]{xcolor}

\title{\textbf{Management in Global Markets\\Chapter 2: Practice Questions}}
\subtitle{Cross-Cultural Competence}
\author{Universidad Complutense de Madrid\\Group A2}
\date{Academic Year 2025-2026}

\begin{document}

\maketitle

\section*{Instructions}
This document contains practice questions covering Chapter 2: Cross-Cultural Competence. Use these questions to test your understanding of cultural dimensions, communication, and cross-cultural management.

\vspace{0.5cm}
\hrule
\vspace{0.5cm}

\section{Culture and Cultural Dimensions}

\subsection{Fill-in-the-Blank Questions}

\begin{enumerate}[leftmargin=*]
    \item \textbf{Culture} can be defined as the set of shared \underline{\hspace{3cm}}, \underline{\hspace{3cm}}, and \underline{\hspace{3cm}} that distinguish one group of people from another.
    
    \item Geert Hofstede identified \underline{\hspace{1cm}} cultural dimensions based on research with over \underline{\hspace{3cm}} individuals from more than \underline{\hspace{2cm}} countries.
    
    \item \textbf{Power Distance} measures the extent to which less powerful members accept and expect that \underline{\hspace{3cm}} is distributed unequally.
    
    \item In \textbf{high power distance} cultures, subordinates are \underline{\hspace{3cm}} for mistakes, while in \textbf{low power distance} cultures, the \underline{\hspace{3cm}} is blamed.
    
    \item \textbf{Individualism} emphasizes "\underline{\hspace{1cm}}" consciousness, while \textbf{Collectivism} emphasizes "\underline{\hspace{1cm}}" consciousness.
    
    \item The \textbf{35/65 Rule} states that \underline{\hspace{2cm}}\% of communication is what you say, while \underline{\hspace{2cm}}\% is how you say it (body language, tone, expressions).
    
    \item \textbf{Cultural Distance} measures how different cultures are from each other across dimensions such as \underline{\hspace{3cm}}, \underline{\hspace{3cm}}, \underline{\hspace{3cm}}, and \underline{\hspace{3cm}}.
    
    \item \textbf{Cultural Empathy} is understanding and appreciating cultural differences because they make sense in \underline{\hspace{4cm}} context.
\end{enumerate}

\subsection{Multiple Choice Questions}

\begin{enumerate}[resume, leftmargin=*]
    \item Which countries have achieved near-zero gender gap according to Hofstede's research?
    \begin{enumerate}[label=\alph*)]
        \item USA and UK
        \item Norway and Sweden
        \item China and India
        \item France and Germany
    \end{enumerate}
    
    \item In high power distance cultures:
    \begin{enumerate}[label=\alph*)]
        \item There is much trust among coworkers
        \item Subordinates are considered equal colleagues
        \item There is not much trust among coworkers
        \item The system is blamed for mistakes
    \end{enumerate}
    
    \item The 35/65 Rule refers to:
    \begin{enumerate}[label=\alph*)]
        \item The ratio of verbal to non-verbal communication
        \item The percentage of cultural differences
        \item The power distance ratio
        \item The individualism-collectivism split
    \end{enumerate}
    
    \item Phase 2 of Culture Shock is characterized by:
    \begin{enumerate}[label=\alph*)]
        \item Excitement and euphoria
        \item Everything feeling weird and annoying (worst phase)
        \item Getting used to the culture
        \item Stable adaptation
    \end{enumerate}
\end{enumerate}

\section{Hofstede's Cultural Dimensions}

\subsection{Short Answer Questions}

\begin{enumerate}[resume, leftmargin=*]
    \item Explain Power Distance and provide examples of high and low power distance countries. How does this affect business practices?
    
    \vspace{4cm}
    
    \item Compare Individualism and Collectivism. Give examples of countries for each and explain how this affects workplace dynamics.
    
    \vspace{4cm}
    
    \item Describe Long-Term vs. Short-Term Orientation. How does this dimension affect business planning and decision-making?
    
    \vspace{4cm}
    
    \item Explain Uncertainty Avoidance and its implications for international business. Provide examples.
    
    \vspace{4cm}
\end{enumerate}

\section{Culture Shock and Adaptation}

\subsection{Application Questions}

\begin{enumerate}[resume, leftmargin=*]
    \item Describe the four phases of Culture Shock. What happens in each phase, and what is the critical decision point?
    
    \vspace{5cm}
    
    \item Explain the concept of Acculturation. What is needed to successfully go through this process?
    
    \vspace{4cm}
    
    \item Why is Cultural Intelligence important for international business? How is it developed?
    
    \vspace{4cm}
    
    \item What is Cultural Empathy, and why is it essential for working across cultures?
    
    \vspace{3cm}
\end{enumerate}

\section{Communication Across Cultures}

\subsection{Case Study Questions}

\begin{enumerate}[resume, leftmargin=*]
    \item Explain the 35/65 Rule of Communication. Why is non-verbal communication so important in international business?
    
    \vspace{4cm}
    
    \item How does Cultural Distance affect business success? Provide examples of small vs. large cultural distance.
    
    \vspace{4cm}
    
    \item Describe the difference between verbal and non-verbal language. Give examples of how misunderstandings can occur.
    
    \vspace{4cm}
\end{enumerate}

\section{Managing Across Cultures}

\subsection{Analysis Questions}

\begin{enumerate}[resume, leftmargin=*]
    \item Compare high and low power distance cultures in terms of:
    \begin{itemize}
        \item How mistakes are handled
        \item Information sharing
        \item Boss-employee relationships
        \item Decision-making processes
    \end{itemize}
    
    \vspace{5cm}
    
    \item Explain why you cannot "escape" culture and communication when doing international business. What can and cannot be changed?
    
    \vspace{4cm}
    
    \item Design a training program for employees going to work in a culturally distant country. What components would you include?
    
    \vspace{5cm}
\end{enumerate}

\section{Synthesis Questions}

\begin{enumerate}[resume, leftmargin=*]
    \item Create a comprehensive comparison table of all six Hofstede dimensions, including:
    \begin{itemize}
        \item Definition
        \item High vs. Low characteristics
        \item Example countries
        \item Business implications
    \end{itemize}
    
    \vspace{8cm}
    
    \item Write an essay (400-500 words) explaining how understanding cultural dimensions can prevent business failures in international markets. Include real examples from the course.
    
    \vspace{10cm}
    
    \item Develop a framework for assessing cultural compatibility between your home country and a target market. Include all relevant dimensions and factors.
    
    \vspace{6cm}
\end{enumerate}

\newpage

\section*{Answer Key}

\textit{Note: Answers should be reviewed with course materials and instructor guidance.}

\subsection{Definitions}
\begin{enumerate}
    \item beliefs, values, norms
    \item 6 (or 5), 100,000, 50
    \item power
    \item blamed, system
    \item I, We
    \item 35, 65
    \item economy, politics, society, geography (or similar)
    \item their cultural
\end{enumerate}

\subsection{Multiple Choice}
\begin{enumerate}[resume]
    \item (b) Norway and Sweden
    \item (c) There is not much trust among coworkers
    \item (a) The ratio of verbal to non-verbal communication
    \item (b) Everything feeling weird and annoying (worst phase)
\end{enumerate}

\end{document}
