\documentclass[11pt,a4paper]{article}
\usepackage[utf8]{inputenc}
\usepackage[margin=1in]{geometry}
\usepackage{amsmath}
\usepackage{amssymb}
\usepackage{enumitem}
\usepackage{tikz}
\usepackage{graphicx}
\usepackage{multirow}
\usepackage{array}
\usepackage{tabularx}
\usepackage[table]{xcolor}

\title{\textbf{Management in Global Markets\\Chapter 1: Practice Questions}}
\subtitle{Globalization, Glocalization, and Slowbalization}
\author{Universidad Complutense de Madrid\\Group A2}
\date{Academic Year 2025-2026}

\begin{document}

\maketitle

\section*{Instructions}
This document contains practice questions covering Chapter 1: Globalization, Glocalization, and Slowbalization. Use these questions to test your understanding and prepare for exams.

\vspace{0.5cm}
\hrule
\vspace{0.5cm}

\section{Definitions and Concepts}

\subsection{Fill-in-the-Blank Questions}

\begin{enumerate}[leftmargin=*]
    \item \textbf{Globalization} is defined as the free movement of \underline{\hspace{3cm}}, \underline{\hspace{3cm}}, and \underline{\hspace{3cm}} across borders.
    
    \item \textbf{Slowbalization} refers to the \underline{\hspace{4cm}} of globalization, characterized by increased \underline{\hspace{3cm}} and reduced cross-border trade.
    
    \item \textbf{Glocalization} is the process of adapting \underline{\hspace{3cm}} products and services to fit \underline{\hspace{3cm}} markets.
    
    \item The three main winners of globalization are \underline{\hspace{2cm}}, \underline{\hspace{2cm}}, and \underline{\hspace{3cm}}.
    
    \item \textbf{Economic Complexity} is the ability to transform \underline{\hspace{4cm}} into \underline{\hspace{4cm}} products using technology, innovation, and R\&D.
    
    \item China is the world's biggest \underline{\hspace{3cm}}, while the USA is the world's biggest \underline{\hspace{3cm}}.
    
    \item The \underline{\hspace{2cm}} represents approximately 50\% of global economic activity and consists of 20 countries with the highest GDP.
    
    \item \textbf{Localization} is necessary before \underline{\hspace{5cm}} can occur.
\end{enumerate}

\subsection{Multiple Choice Questions}

\begin{enumerate}[resume, leftmargin=*]
    \item Which of the following best describes "slowbalization"?
    \begin{enumerate}[label=\alph*)]
        \item The acceleration of global trade
        \item The slowing down of globalization with increased protectionism
        \item The complete reversal of globalization
        \item The digitalization of global markets
    \end{enumerate}
    
    \item What is the primary difference between globalization and economic complexity?
    \begin{enumerate}[label=\alph*)]
        \item Globalization focuses on trade, while economic complexity focuses on production capacity
        \item Economic complexity requires innovation and technology to transform raw materials
        \item Globalization is about volume, economic complexity is about value
        \item All of the above
    \end{enumerate}
    
    \item Which country has twice the oil reserves of Saudi Arabia but remains economically poor?
    \begin{enumerate}[label=\alph*)]
        \item Iran
        \item Iraq
        \item Venezuela
        \item Russia
    \end{enumerate}
    
    \item By 2032, Europe is projected to have:
    \begin{enumerate}[label=\alph*)]
        \item 400 million people, with 60\% over age 60
        \item 500 million people, with 50\% over age 60
        \item 300 million people, with 70\% over age 60
        \item 600 million people, with 40\% over age 60
    \end{enumerate}
\end{enumerate}

\section{Global Players and Demographics}

\subsection{Short Answer Questions}

\begin{enumerate}[resume, leftmargin=*]
    \item Explain why China and India are considered "globalization winners." List at least three reasons.
    
    \vspace{3cm}
    
    \item Describe the demographic shifts expected over the next 50-60 years. What are the implications for global business?
    
    \vspace{4cm}
    
    \item Compare and contrast the economic situations of Venezuela and Saudi Arabia. What lessons can be learned?
    
    \vspace{4cm}
\end{enumerate}

\section{Economic Complexity and Production}

\subsection{Application Questions}

\begin{enumerate}[resume, leftmargin=*]
    \item List the top countries in terms of economic complexity. What do they have in common?
    
    \vspace{3cm}
    
    \item Why does Africa lag behind in economic complexity? What factors contribute to this?
    
    \vspace{3cm}
    
    \item Explain the relationship between GDP, production, and exports. Use examples from the top three countries.
    
    \vspace{4cm}
\end{enumerate}

\section{Trade and Tariffs}

\subsection{Case Study Questions}

\begin{enumerate}[resume, leftmargin=*]
    \item Describe the US trade policy strategy regarding China. What is the philosophy behind this approach?
    
    \vspace{3cm}
    
    \item Explain the trade conflict between France and Brazil regarding meat and food production. What are the key issues?
    
    \vspace{3cm}
    
    \item What is the purpose of tariffs in international trade? Provide examples from the course material.
    
    \vspace{3cm}
\end{enumerate}

\section{The G20 and Global Economy}

\subsection{Analysis Questions}

\begin{enumerate}[resume, leftmargin=*]
    \item Why are there 20 countries in the G20? What criteria determine membership?
    
    \vspace{3cm}
    
    \item Explain why Spain is part of the G20 despite being a smaller country. What factors contribute to its inclusion?
    
    \vspace{3cm}
    
    \item How does the G20 relate to the G7 and G5? What is their combined significance?
    
    \vspace{3cm}
\end{enumerate}

\section{Key Takeaways}

\subsection{Synthesis Questions}

\begin{enumerate}[resume, leftmargin=*]
    \item Create a comprehensive comparison table of Globalization vs. Slowbalization vs. Glocalization, including characteristics, examples, and implications.
    
    \vspace{6cm}
    
    \item Write an essay (300-400 words) explaining how demographic shifts will reshape the global economy over the next 50 years. Include specific examples and implications for business.
    
    \vspace{8cm}
    
    \item Design a framework for evaluating a country's potential as a market for international business expansion. Include economic, demographic, and political factors.
    
    \vspace{6cm}
\end{enumerate}

\newpage

\section*{Answer Key}

\textit{Note: Answers should be reviewed with course materials and instructor guidance.}

\subsection{Definitions}
\begin{enumerate}
    \item goods, services, capital
    \item slowing down, protectionism
    \item global, local
    \item China, India, Indonesia
    \item raw materials, complex
    \item exporter, importer
    \item G20
    \item internationalization
\end{enumerate}

\subsection{Multiple Choice}
\begin{enumerate}[resume]
    \item (b) The slowing down of globalization with increased protectionism
    \item (d) All of the above
    \item (c) Venezuela
    \item (a) 400 million people, with 60\% over age 60
\end{enumerate}

\end{document}
